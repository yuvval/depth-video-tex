\documentclass{article}
\usepackage[utf8]{inputenc}
\usepackage{amsmath}

\newcommand{\norm}[1]{\left\lVert #1 \right\rVert}
\newcommand\mat[1]{{#1}}
\newcommand{\T}{{}^\mathsf{T}}
\newcommand{\eqdef}{\doteq}
\renewcommand\vec[1]{\mathbf{#1}}

\title{depth video tex}
\author{yuval }
\date{September 2015}

\begin{document}

\maketitle

\section{Intra Frame Constraints}

Given an image and it's depth estimation, we define a quadratic loss:

\begin{equation}
\sum_{i,j}{w_{i,j} (x_i  - x_j)^2} + \sum_{k}{(x_k - d_k)^2}
\label{objective-elem}
\end{equation}
where, $d_k$ represents the depth estimation per pixel $k$ and $w_{i,j}$ equals
\begin{equation}
w_{i,j} = exp(-1/\rho_{_{E}} \norm{edge\_intensity_{i,j}}^2)
\label{weights_intra}
\end{equation}
if pixels $i,j$ are neighbouring pixels, or $0$ otherwise. 

${w_{i,j} (x_i  - x_j)^2}$ is a smoothness term that encourages neighbouring pixels to have a similar depth if there is no edge between them.
$\rho_{_{E}}$ controls the balance between the given depth estimation and the smoothness term.

The first (binary) term of \eqref{objective-elem} can be formulated in matrix notation as:
\begin{equation}
x\T \mat{W^{b}} x
\end{equation}
where 
\begin{align}
\mat{W^{b}} \eqdef \mat{W^{r}} + \mat{W^{c}} - 2\mat{W}\\
\mat{W}_{i,j} \eqdef w_{i,j}
\end{align}
and $\mat{W^{r}}$, $\mat{W^{c}}$ are diagonal matrices that their diagonal elements equal to:
\begin{align}
\mat{W^{r}}_{i,i} \eqdef \sum_{j}{w_{i,j}}\\ \mat{W^{c}}_{j,j} \eqdef \sum_{i}{w_{i,j}} \end{align}

The second (unary) term of \eqref{objective-elem} can be formulated in matrix notation as:
\begin{equation}
x\T \mat{I} x + -2 \vec{d}\T x + \vec{d}\T \vec{d}
\end{equation}
where \mat{I} is a square identity matrix and $\vec{d}$ is the vector of given estimated depths.

Wrapping it up, \eqref{objective-elem} results with
\begin{equation}
x\T \mat{W}^{q} x - 2 \vec{d}\T x + \vec{d}\T \vec{d}
\end{equation}
where $\mat{W}^{q} \eqdef \mat{W^{r}} + \mat{W^{c}} - 2\mat{W} + \mat{I} $




\section{Inter Frame Constraints}

We now add constraints between two consecutive frames $\{t, t+1\}$ of depth estimations. Note that we index all pixels depth values over space (i,j) and time (k) to a 1D vector array. 

For each frame we evaluate the optical flow (OF) of each pixel to the next one. Then, we can represent the flow with the vector array indices. Pixel $m$ (originated from frame $t$), now is mapped to pixel $n$. For each such $m,n$ transition between consecutive frames, we introduce a weight $w_{m,n}$ which depends on the OF magnitude
\begin{equation}
w_{m,n} = exp(-1/\rho_{_{OF}} \norm{OF\_magnitude_{m,n}}^2)
\label{weights_inter}
\end{equation}
if there is an OF between pixels $m,n$, or $0$ otherwise. 
$\rho_{_{OF}}$ controls the balance between the given depth estimation and the OF smoothness term.

We extend \eqref{objective-elem} to account for consecutive frames, such that \eqref{weights_intra} is evaluated for pixels that reside within each of the frames and \eqref{weights_inter} is evaluated for pixels that reside between consecutive frames.

\end{document}

